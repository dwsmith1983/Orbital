\hypersetup{pageanchor = false}
\chapter{Parabolic Trajectories}
\label{parab-traj}

A parabolic orbit is when the eccentricity, \(e = 1\), and \(\mathcal{E} = 0\).
Since we are dealing with a parabola, we can relate the directrix with the
semi-latus rectum, \(p\), and \(r_p\).
That is, \(r_p = \frac{p}{2}\Rightarrow p = 2r_p\).
Since \(\mathcal{E} = 0 = \frac{v^2}{2} - \frac{\mu}{r}\),
\(v = \sqrt{\frac{2\mu}{r}}\) which is the speed of a parabolic path.
\begin{figure}
  \centering
  \includestandalone[mode = image]{geometryparabola}
  \caption[Geometry of a Parabola]{The geometry of a parabolic trajectory.}
  \label{geometryparabola}
\end{figure}
\noindent
For a parabolic trajectory, we can write the flight angle as:
\[
\tan\gamma = \frac{\sin\nu}{1 + \cos\nu}.\eqnumtag
\label{flightpathangleparabola}
\]
Using the following identities, we can simplify the right hand side of \cref{flightpathangleparabola}.
\begin{align} 
  \sin(2\theta) & = 2\sin(\theta)\cos(\theta)\\
  \cos(2\theta) & = 2\cos^2(\theta) - 1
  \label{doubleangle}
\end{align}
That is, \(\tan\gamma = \tan\frac{\nu}{2}\) so \(\gamma = \frac{\nu}{2}\).
\begin{examples}{Parabolic Trajectory}{parabolictrajectory}
  What minimum velocity, relative to the Earth, is required to escape the solar  
  system along a parabolic path from Earth orbit?
  \par\medskip
  \begin{minipage}{\linewidth}
    \centering
    \includestandalone[mode = image]{parabolictrajearth}
    \captionof{figure}[Parabolic Trajectory-Earth]
    {A parabolic trajectory from Earth.}
    \label{parabolictrajearth}
  \end{minipage}
  \par\medskip
  The velocity of the Earth is
  \(v_{\Earth} = \sqrt{\frac{\mu_{\Sun}}{2a_{\Earth}}} = 29.784\) km/s where
  \(\mu_{\Sun} = 132712440018\) and \(a_{\Earth} = 149.6\times 10^6\).
  Now \(v_{\text{esc}} = \sqrt{2}v_{\Earth} = 42.1212\) km/s.
  \begin{align*} 
    v_{\text{esc}} & = v_{\text{rel}} + v_{\Earth}\\
    v_{\text{rel}} & = v_{\text{esc}} - v_{\Earth}\\
    & = 42.1212 - 29.784\\
    & = 12.337\text{km}/\text{s}
  \end{align*}
\end{examples}
%%% Local Variables:
%%% mode: latex
%%% TeX-master: t
%%% End:
