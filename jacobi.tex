\hypersetup{pageanchor = false}
\section{Jacobi Constant} 
\label{jacobi-constant}

In order to find the Jacobi constant, we need to turn back to our our original
system of ODEs:
\begin{align}
  \ddot{x} - 2\varOmega\dot{y} - \varOmega^2x
  & =   -\frac{\mu_1}{r_1^3}(x + \pi_2r_{12}) -
    \frac{\mu_2}{r_2^3}(x - \pi_1r_{12})\label{xjacobi}\\
  \ddot{y} + 2\varOmega\dot{x} - \varOmega^2y
  & =   -\frac{\mu_1}{r_1^3}y - \frac{\mu_2}{r_2^3}y\label{yjacobi}\\
  \ddot{z} & = -\frac{\mu_1}{r_1^3}z - \frac{\mu_2}{r_2^3}z\label{zjacobi}
\end{align}
Let's multiple \cref{xjacobi} by \(\dot{x}\), \cref{yjacobi} by \(\dot{y}\), and \cref{zjacobi} by \(\dot{z}\).
\begin{align}
  \dot{x}\ddot{x} - 2\varOmega\dot{x}\dot{y} - \varOmega^2\dot{x}x
  & = -\dot{x}\frac{\mu_1}{r_1^3}(x + \pi_2r_{12}) -
    \dot{x}\frac{\mu_2}{r_2^3}(x - \pi_1r_{12})\label{x2jacobi}\\
  \dot{y}\ddot{y} + 2\varOmega\dot{y}\dot{x} - \varOmega^2\dot{y}y
  & = -\frac{\mu_1}{r_1^3}\dot{y}y - \frac{\mu_2}{r_2^3}\dot{y}y
    \label{y2jacobi}\\
  \dot{z}\ddot{z}
  & = -\frac{\mu_1}{r_1^3}\dot{z}z - \frac{\mu_2}{r_2^3}\dot{z}z
    \label{z2jacobi}
\end{align}
Now we can add the \cref{x2jacobi,y2jacobi,z2jacobi} together.
\begin{align*}
  \dot{x}\ddot{x} + \dot{y}\ddot{y} + \dot{z}\ddot{z} -
  \varOmega^2(\dot{x}\ddot{x} + \dot{y}\ddot{y})
  & = -\Bigl(\frac{\mu_1}{r_1^3} + \frac{\mu_2}{r_2^3}\Bigr)(\dot{x}x +
    \dot{y}y + \dot{z}z) + \Bigl(\frac{\pi_1\mu_2}{r_2^3}
    - \frac{\pi_2\mu_1}{r_1^3}\Bigr)r_{12}\dot{x}\\
  \frac{1}{2}\frac{d}{dt}(v^2) - \frac{1}{2}\varOmega^2\frac{d}{dt}(r^2)
  & =  -\frac{\mu_1}{r_1^3}[(x + \pi_2r_{12})\dot{x} + \dot{y}y + \dot{z}z]
    - \frac{\mu_2}{r_2^3}[(x - \pi_1r_{12})\dot{x} + \dot{y}y + \dot{z}z]\\
  \frac{d}{dt}\Bigl[\frac{v^2}{2} - \frac{1}{2}\varOmega^2r^2
  - \frac{\mu_1}{r_1} - \frac{\mu_2}{r_2}\Bigr]
  & = 0\\
  \frac{v^2}{2} - \frac{1}{2}\varOmega^2r^2 - \frac{\mu_1}{r_1}
  - \frac{\mu_2}{r_2}
  & = C\\
  v^2
  & = \varOmega^2r^2 + \frac{2\mu_1}{r_1} + \frac{2\mu_2}{r_2} + 2C
\end{align*}
where
\begin{align*}
  r^2 & = x^2 + y^2\\
  \dot{r}r & = \dot{x}x + \dot{y}y\\
  r_1^2 & = (x + \pi_2r_{12})^2 + y^2 + z^2\\
  \dot{r}_1
      & =  \frac{1}{r_1}\Bigl[\dot{x}(x + \pi_2r_{12}) + \dot{y}y
        + \dot{z}z\Bigr]\\
  r_2^2 & = (x - \pi_1r_{12})^2 + y^2 + z^2\\
  \dot{r}_2
      & =  \frac{1}{r_2}\Bigl[\dot{x}(x - \pi_1r_{12}) + \dot{y}y
        + \dot{z}z\Bigr]
\end{align*}
Since \(v^2 \geq 0\),
\(\varOmega^2r^2 + \frac{2\mu_1}{r_1} + \frac{2\mu_2}{r_2} + 2C \geq 0\) too.
When \(\varOmega^2r^2 + \frac{2\mu_1}{r_1} + \frac{2\mu_2}{r_2} + 2C = 0\), we
have zero velocity surfaces.
\begin{examples}{Lagrange Point \(L_4\)}{lagrangepointl4}
  Using a scientific software package, develop a computer program to compute
  the trajectory of a spacecraft using the restricted three-body equations of
  motion.
  Use this program to design a trajectory from Earth to the Earth-Moon Lagrange
  point \(L_4\) starting at a 200km altitude burnout point.
  The mission design requirement is that the trajectory should take the
  coasting  spacecraft to within 500km of \(L_4\) with a relative speed of no
  more than \(1\) km/s.
  \par\smallskip
  \begin{minipage}{\linewidth}
    \centering
    \includestandalone[mode = image]{earthmoonl4}
    \captionof{figure}[Mission to \(L_4\)]{Mission to Earth-Moon \(L_4\)
      point.}
    \label{earthmoonl4}
  \end{minipage}
  \par\smallskip
  First, we need to determine the location of \(L_4\).
  Since \(L_4\) is lying in the the \(xy\)-plane, the \(z\)-coordinate is 0.
  The center of gravity is located 1707km inside the Earth.
  Therefore, the center of Earth is at \((-4671,0,0)\) km.
  To find the \(y\)-coordinate, we have
  \(\frac{\sqrt{3}}{2}384400 = 332900.1652\) km.
  Using Pythagoras's Theorem, \(x = \sqrt{384400^2 - 332900^2}\approx 187529\)
  km.
  That is, \(L_4\) is at \((187529, 332900, 0)\).
  With this information, we can find Jacobi's constant
  \begin{align*}
    \varOmega^2(x_{L_4}^2 + y_{L_4}^2) + \frac{2\mu_1}{r_1} +
    \frac{2\mu_2}{r_2} + 2C & < 1\\
    C & < -1.06824
  \end{align*}
  where
  \(\varOmega = \sqrt{\frac{6.67259\times 10^{-20}(m_{\text{e}} +          
      m_{\text{m}})}{r_{12}}}\),
  \(\mu_1 = 6.67259\times 10^{-20}m_{\text{e}}\),   
  \(\mu_2 = 6.67259\times 10^{-20}m_{\text{m}}\), and
  \(r_1 = r_2 = r_{12} = 384400\).
  To arrive at \(L_4\) with a relative speed of 0, \(C = -1.56824\).
  Since the true anomaly wasn't specified, I let \(\nu = -\frac{\pi}{4}\).
  So the position of the spacecrafts burnout will be
  \(\mathbf{r}_s = \langle -19.3098,-4651.35,0\rangle\).
  For our trajectory, let \(C = -1.21\).
  At this location and this instance, our velocity will be \(v_{b_o}\).
  \begin{align*}
  v_{b_o} & = \sqrt{\varOmega^2(x^2 + y^2) + \frac{2\mu_1}{\sqrt{(x
            + \pi_2r_{12})^2 + y^2}} + \frac{2\mu_2}{\sqrt{(x - \pi_2r_{12})^2
            + y^2}} + 2(-1.21)}\\
          & = 10.8994415375\text{ km}/\text{s}
  \end{align*}
  where \(x\) and \(y\) are the components of \(\mathbf{r}_s\).
  Let the flight path angle be \(\gamma = -14.02^{\circ}\).
  The \(x\) and \(y\) components on \(\mathbf{v}_{0}\) are
  \begin{align*}
    v_x & = v_{b_o}(\sin\gamma\cos\nu - \sin\nu\cos\gamma)\\
        & = 7.76974\\
    v_y & = v_{b_o}(\sin\gamma\sin\nu + \cos\nu\cos\gamma)\\
        & = 7.64389
  \end{align*}
  Then we can write \(\mathbf{v}_0 = \langle 7.76974,7.64389,0\rangle\).
  \par\smallskip
  \begin{minipage}{\linewidth}
    \centering
    \includegraphics[width = 3.5in, height = 3.5in, trim = 20bp 20bp 55bp 40bp,
    clip]{example6notes}
    \captionof{figure}[Simulation to \(L_4\)]
    {Flight path with the given conditions to \(L_4\) of the Earth-Moon
      system.}
    \label{launchtol4earthmoon}
  \end{minipage}
  \par\smallskip
  At the position \(\langle 187529.4904, 332900.6847, 0\rangle\), the
  associated   velocity vector is \(\langle 0.6428, -0.5506, 0\rangle\).
  That is, the speed at this location is
  \[
  \sqrt{0.6428^2 + 0.5506^2} = 0.846455944534\text{ km}/\text{s} <
  1\text{ km}/\text{s},
  \]
  and the distance from \(L_4\) is
  \begin{align*}
    \sqrt{(187529.4904 - 187529)^2 + (332900.6847 - 332900)^2}
    & = 0.846455944534\text{ km}\\
    & < 500\text{ km}.
  \end{align*}
  For the exact calculations, see example6notes.py.
\end{examples}
%%% Local Variables:
%%% mode: latex
%%% TeX-master: t
%%% End:
