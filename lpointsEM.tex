\hypersetup{pageanchor = false}
\section{Lagrange Points for the Earth-Moon System}
\label{lagr-points-earth}

\begin{examples}{Lagrange Points \(L_1\), \(L_2\), and \(L_3\)}{lagrangepoints}
  Refer to the Python file example5notes.py for the calculation of \(L_1\),
  \(L_2\), and \(L_3\).
  See \cref{numericallyfindlagrangepoints} for a plot of the \(f(\xi)\).
  \par\smallskip
  \begin{minipage}{\linewidth}
    \centering
    \includegraphics[width = 3.5in, height = 3.5in, trim = 30bp 30bp 30bp 35bp,
    clip]{example5notes}
    \captionof{figure}[Lagrange Points \(1\), \(2\), and \(3\)]
    {The \(\xi\) intercepts of \(f(\xi)\).}
    \label{numericallyfindlagrangepoints}
  \end{minipage}
  \par\smallskip
  To find \(L_1\), \(L_2\), and \(L_3\) of the Earth-Moon system, we simply
  multiple \(\xi\) times \(r_{12}\).
  \begin{align*} 
    L_1 & = 0.8369\cdot r_{12}\\
        & = 321709.713544\\
    L_2 & = 1.15568\cdot r_{12}\\
        & = 444244.584579\\
    L_3 & = -1.005062\cdot r_{12}\\
        & = -386346.120068
  \end{align*}
\end{examples}
%%% Local Variables:
%%% mode: latex
%%% TeX-master: t
%%% End:
