\chapter{Restricted 3-Body Problem}
\label{restricted-3-body}

\begin{figure}
  \centering
  \includestandalone[mode = image]{restricted3bodydiagram}
  \caption[3 Body Diagram]{Diagram of the restricted 3 body problem.}
  \label{restricted3bodydiagram}
\end{figure}
\begin{alignat*}{4} 
  r_{12} & = \sqrt{(x_1 - x_2)^2} & \qquad & x_1
  &&{}= \text{Is the } x \text{ location of } m_1\\
  x_2 & = x_1 + r_{12} & & &&{}\phantom{=}
  \text{relative to the center of gravity.}\\
  x_1 & = \frac{-m_2}{m_1 + m_2}r_{12} & & \pi_1 &&{}=
  \frac{m_1}{m_1 + m_2}\\
  x_2 & = \frac{m_1}{m_1 + m_2}r_{12} & & \pi_2 &&{}=
  \frac{m_2}{m_1 + m_2}\\
  0 & = m_1x_1 + m_2x_2
\end{alignat*}
We can describe the position of \(m\) as
\(\mathbf{r} = x\hat{\mathbf{i}} + y\hat{\mathbf{j}} + z\hat{\mathbf{k}}\) in relation to the center of gravity, i.e., the origin.
\begin{align} 
  \mathbf{r}_1
  & =  (x - x_1)\hat{\mathbf{i}} + y\hat{\mathbf{j}} +
    z\hat{\mathbf{k}}\notag\\
  & = (x + \pi_2r_{12})\hat{\mathbf{i}} + y\hat{\mathbf{j}}
    + z\hat{\mathbf{k}}\label{position1}\\
  \mathbf{r}_2
  & =  (x - x_2)\hat{\mathbf{i}} + y\hat{\mathbf{j}} +
    z\hat{\mathbf{k}}\notag\\
  & = (x - \pi_1r_{12})\hat{\mathbf{i}} + y\hat{\mathbf{j}}
    + z\hat{\mathbf{k}}\label{position2}
\end{align}
Let's define the absolute acceleration where \(\omega\) is the initial angular
velocity which is constant.
Then \(\omega = \frac{2\pi}{T}\).
\[ 
\ddot{\mathbf{r}}_{\text{abs}}
= \mathbf{a}_{\text{rel}} + \mathbf{a}_{\text{CG}}
+ \mathbf{\varOmega}\times(\mathbf{\varOmega}\times\mathbf{r})
+ \dot{\mathbf{\varOmega}}\times\mathbf{r}
+ 2\mathbf{\varOmega}\times\mathbf{v}_{\text{rel}}
\eqnumtag\label{absacceleration}
\]
where
\begin{alignat*}{4}
  \mathbf{a}_{\text{rel}}
  & = \text{Rectilinear acceleration relative to the frame} & \quad &
  \mathbf{\varOmega}\times(\mathbf{\varOmega}\times\mathbf{r}) &&{}= 
  \text{Centripetal acceleration}\\
  2\mathbf{\varOmega}\times\mathbf{v}_{\text{rel}} & =
  \text{Coriolis acceleration}
\end{alignat*}
Since the velocity of the center of gravity is constant,
\(\mathbf{a}_{\text{CG}} = 0\), and \(\dot{\mathbf{\varOmega}} = 0\) since the
angular velocity of a circular orbit is constant.
Therefore, \cref{absacceleration} becomes:
\[ 
\ddot{\mathbf{r}} = \mathbf{a}_{\text{rel}}
+ \mathbf{\varOmega}\times(\mathbf{\varOmega}\times\mathbf{r})
+ 2\mathbf{\varOmega}\times\mathbf{v}_{\text{rel}}
\eqnumtag\label{modabsaccel}
\]
where
\begin{align} 
  \mathbf{\varOmega} & = \varOmega\hat{\mathbf{k}}\label{omega}\\
  \mathbf{r} & = x\hat{\mathbf{i}} + y\hat{\mathbf{j}}
               + z\hat{\mathbf{k}}\label{position}\\
  \dot{\mathbf{r}} & = \mathbf{v}_{\text{CG}}
                     + \mathbf{\varOmega}\times\mathbf{r}
                     + \mathbf{v}_{\text{rel}}\label{velocity}\\
  \mathbf{v}_{\text{rel}} & = \dot{x}\hat{\mathbf{i}}
                            + \dot{y}\hat{\mathbf{j}}
                            + \dot{z}\hat{\mathbf{k}}\label{relvel}\\
  \mathbf{a}_{\text{rel}} & = \ddot{x}\hat{\mathbf{i}}
                            + \ddot{y}\hat{\mathbf{j}}
                            + \ddot{z}\hat{\mathbf{k}}\label{relabsvel}
\end{align}
After substituting \cref{omega}, \cref{position}, \cref{relvel}, and
\cref{relabsvel} into \cref{modabsaccel}, we obtain
\[
\ddot{\mathbf{r}} =
\left(\ddot{x} - 2\varOmega\dot{y} - \varOmega^2x\right)\hat{\mathbf{i}}
+ \left(\ddot{y} + 2\varOmega\dot{x} - \varOmega^2y\right)\hat{\mathbf{j}}
+ \ddot{z}\hat{\mathbf{k}}.
\eqnumtag\label{simplifiedabsaccel}
\]
Newton's \(2^{\text{nd}}\) Law of Motion is
\(m\mathbf{a} = \mathbf{F}_1 + \mathbf{F}_2\) where
\(\mathbf{F}_1 = -\frac{Gm_1m}{r_1^3}\mathbf{r}_1\) and
\(\mathbf{F}_2 = -\frac{Gm_2m}{r_2^3}\mathbf{r}_2\).
Let \(\mu_1 = Gm_1\) and \(\mu_2 = Gm_2\).
\begin{align} 
  m\mathbf{a} & = \mathbf{F}_1 + \mathbf{F}_2 & \notag\\
  m\mathbf{a} & = -\frac{m\mu_1}{r_1^3}\mathbf{r}_1
                - \frac{m\mu_2}{r_2^3}\mathbf{r}_2 & \notag\\
  \mathbf{a} & = -\frac{\mu_1}{r_1^3}\mathbf{r}_1
               - \frac{\mu_2}{r_2^3}\mathbf{r}_2 & \notag\\
  \bigl(\ddot{x} - 2\varOmega\dot{y} - \varOmega^2x\bigr)\hat{\mathbf{i}}
  + \bigl(\ddot{y} + 2\varOmega\dot{x} - \varOmega^2y\bigr)\hat{\mathbf{j}}
  + \ddot{z}\hat{\mathbf{k}}
              & = -\frac{\mu_1}{r_1^3}\mathbf{r}_1
                - \frac{\mu_2}{r_2^3}\mathbf{r}_2 & \notag\\
  \bigl(\ddot{x} - 2\varOmega\dot{y} - \varOmega^2x\bigr)\hat{\mathbf{i}}
  + \bigl(\ddot{y} + 2\varOmega\dot{x} - \varOmega^2y\bigr)\hat{\mathbf{j}}
  + \ddot{z}\hat{\mathbf{k}}
              & =
                \begin{aligned}
                  - & \frac{\mu_1}{r_1^3}\Bigl[(x +
                  \pi_2r_{12})\hat{\mathbf{i}} + \hat{\mathbf{j}} +
                  \hat{\mathbf{k}}\Bigr]\\
                  - & \frac{\mu_2}{r_2^3}\Bigl[(x -
                  \pi_1r_{12})\hat{\mathbf{i}} + \hat{\mathbf{j}} +
                  \hat{\mathbf{k}}\Bigr]
                \end{aligned}\label{absaccelfurthersolved}
\end{align}
Now all we have to do is equate the coefficients.
\begin{align}
  \ddot{x} - 2\varOmega\dot{y} - \varOmega^2x
  & = -\frac{\mu_1}{r_1^3}(x + \pi_2r_{12}) - \frac{\mu_2}{r_2^3}(x
    - \pi_1r_{12})\\
  \ddot{y} + 2\varOmega\dot{x} - \varOmega^2y
  & = -\frac{\mu_1}{r_1^3}y - \frac{\mu_2}{r_2^3}y\\
  \ddot{z}
  & = -\frac{\mu_1}{r_1^3}z - \frac{\mu_2}{r_2^3}z
\end{align}
We now have system of nonlinear ODEs.
The standard approach is to start by identifying the fixed points (Lagrange
points).
The Lagrange points occur when
\(\dot{x} = \dot{y} = \dot{z} = \ddot{x} = \ddot{y} = \ddot{z} = 0\) which
brings us to the following system of ODEs.
\begin{align*}
  -\varOmega^2x & = -\frac{\mu_1}{r_1^3}(x + \pi_2r_{12}) -
                   \frac{\mu_2}{r_2^3}(x - \pi_1r_{12})\\
  -\varOmega^2y & = -\frac{\mu_1}{r_1^3}y - \frac{\mu_2}{r_2^3}y\\
  0 & = -\frac{\mu_1}{r_1^3}z - \frac{\mu_2}{r_2^3}z
\end{align*}
We can clearly see from the last equation that \(z = 0\).
We can simplify the last two ODEs further by using the relations
\(\varOmega = \sqrt{\frac{\mu}{r_{12}^3}}\), \(\pi = 1 - \pi_2\),
\(\pi_1 = \frac{\mu_1}{\mu}\), and \(\pi_2 = \frac{\mu_2}{\mu}\).
\begin{align}
  -\frac{\mu}{r_{12}^3}x
  & = -\frac{\mu_1}{r_1^3}(x + \pi_2r_{12})
    - \frac{\mu_2}{r_2^3}(x - \pi_1r_{12})\notag\\
  \frac{x}{r_{12}^3}
  & = \frac{\mu_1}{\mu}\frac{1}{r_1^3}(x + \pi_2r_{12})
    + \frac{\mu_2}{\mu}\frac{1}{r_2^3}[x - (1 - \pi_2)r_{12}]\notag\\
  \frac{x}{r_{12}^3}
  & = \pi_1\frac{1}{r_1^3}(x + \pi_2r_{12})
    + \pi_2\frac{1}{r_2^3}[x  - r_{12} + \pi_2r_{12}]\notag\\
  \frac{x}{r_{12}^3}
  & = \frac{(1 - \pi_2)}{r_1^3}(x + \pi_2r_{12})
    + \frac{\pi_2}{r_2^3}[x  - r_{12} + \pi_2r_{12}]\label{xode}\\
  -\frac{\mu}{r_{12}^3}y
  & = -\frac{\mu_1}{r_1^3}y - \frac{\mu_2}{r_2^3}y\notag\\
  \frac{1}{r_{12}^3}
  & = \frac{\mu_1}{\mu}\frac{1}{r_1^3} +
    \frac{\mu_2}{\mu}\frac{1}{r_2^3}\notag\\
  \frac{1}{r_{12}^3}
  & = \frac{\pi_1}{r_1^3} + \frac{\pi_2}{r_2^3}\notag\\
  \frac{1}{r_{12}^3}
  & = \frac{1 - \pi_2}{r_1^3} + \frac{\pi_2}{r_2^3}\label{yode}
\end{align}
We have ended up with the following two \cref{xode,yode}.
We can solve the equations simultaneously.
\[
\begin{bmatrix}
  (1 - \pi_2)(x + \pi_2r_{12}) & \pi_2(x  - r_{12} + \pi_2r_{12}) & - x\\
  1 - \pi_2 & \pi_2 & - 1
\end{bmatrix}
\overbrace{\Rightarrow}^{\text{rref}}
\begin{bmatrix}
  1 & 0 & -1\\
  0 & 1 & -1
\end{bmatrix}
\eqnumtag\label{linearsystem}
\]
Finally, we have that \(\frac{1}{r_1^3} = \frac{1}{r_{12}^3}\) and
\(\frac{1}{r_2^3} = \frac{1}{r_{12}^3}\) so
\begin{align*}
  \frac{1}{r_1^3} & = \frac{1}{r_2^3} = \frac{1}{r_{12}^3}\\
  r_1 & = r_2 = r_{12}.
\end{align*}
Recall \cref{position1,position2} which we can now rewrite with by using the
previously obtained information: \(z = 0\), \(r_1 = r_2 = r_{12}\), and
\(\pi_1 = 1 - \pi_2\).
\begin{align}
  \mathbf{r}_1
  & = (x + \pi_2r_{12})\hat{\mathbf{i}} + y\hat{\mathbf{j}}\notag\\
  r_1
  & = \sqrt{(x + \pi_2r_{12})^2 + y^2}\notag\\
  r_1^2
  & = (x + \pi_2r_{12})^2 + y^2\notag\\
  r_{12}^2
  & = (x + \pi_2r_{12})^2 + y^2\label{r12one}\\
  \mathbf{r}_2
  & = (x - \pi_1r_{12})\hat{\mathbf{i}} + y\hat{\mathbf{j}}\notag\\
  r_{12}^2 & = (x + \pi_2r_{12} - r_{12})^2 + y^2\label{r12two}
\end{align}
Next we need to set the two \cref{r12one,r12two} equal to each other. 
However, remember that the solution to a square root involves both plus and
minus.
If we take the positive solution, we will end up \(r_{12} = 0\) which is
certainly not the case.
\begin{align} 
  x + \pi_2r_{12} - r_{12} & = -x - \pi_2r_{12}\notag\\
  x & = \frac{r_{12}}{2} - \pi_2r_{12}\label{xsolnsec5}
\end{align}
Lastly,
\(r_{12}^2 = (x + \pi_2r_{12} - r_{12})^2 + y^2\Rightarrow y
= \pm\frac{r_{12}\sqrt{3}}{2}\).
We now have 2 Lagrange points, namely \(L_4\) and \(L_5\).
\[ 
L_4 {}:{} \Bigl(\frac{r_{12}}{2} - \pi_2r_{12}, \frac{r_{12}\sqrt{3}}{2},
0\Bigr)
\quad\text{and}\quad
L_5 {}:{} \Bigl(\frac{r_{12}}{2} - \pi_2r_{12}, - \frac{r_{12}\sqrt{3}}{2},
0\Bigr)
\]
Since \(r_1 = r_2 = r_{12}\), these points form 2 equilateral triangles with
the \(m_1\) and \(m_2\).
\begin{figure}
  \centering
  \includestandalone[mode = image]{lagrangepoints}
  \caption[\(L_4\) and \(L_5\) Lagrange Points]
  {\(L_4\) and \(L_5\) Lagrange points of a 2 body system.}
  \label{lagrangepoints}
\end{figure}
To find the remaining Lagrange points, we let \(y = z = 0\).
Then \(\mathbf{r}_1 = (x + \pi_2r_{12})\hat{\mathbf{i}}\) and
\(\mathbf{r}_2 = (x - \pi_1r_{12})\hat{\mathbf{i}}\).
Let \(\xi = \frac{x}{r_{12}}\).
\begin{align*}
  -\varOmega^2x & = -\frac{\mu_1}{r_1^3}(x + \pi_2r_{12})
                  - \frac{\mu_2}{r_2^3}(x + \pi_2r_{12} - r_{12})\\
  -\frac{\mu}{r_{12}^3}x & = -\frac{\mu_1}{|x + \pi_2r_{12}|^3}(x
                           + \pi_2r_{12}) - \frac{\mu_2}{|x + \pi_2r_{12}
                           - r_{12}|^3}(x + \pi_2r_{12} - r_{12})\\
  0 & = \frac{\pi_1}{|x + \pi_2r_{12}|^3}(x + \pi_2r_{12}) +
      \frac{\pi_2}{|x + \pi_2r_{12} - r_{12}|^3}(x + \pi_2r_{12} - r_{12})
      - \frac{x}{r_{12}^3}\\
  0 & = \frac{(1 - \pi_2)}{|x + \pi_2r_{12}|^3}(x + \pi_2r_{12})
      + \frac{\pi_2}{|x + \pi_2r_{12} - r_{12}|^3}(x + \pi_2r_{12} - r_{12})
      - \frac{x}{r_{12}^3}\\
  0 & = \frac{(1 - \pi_2)}{|r_{12}\xi + \pi_2r_{12}|^3}(r_{12}\xi +
      \pi_2r_{12}) + \frac{\pi_2}{|r_{12}\xi + \pi_2r_{12}
      - r_{12}|^3}(r_{12}\xi + \pi_2r_{12} - r_{12}) -
      \frac{r_{12}\xi}{r_{12}^3}\\
  0 & = \frac{(1 - \pi_2)}{r_{12}^2|\xi + \pi_2|^3}(\xi + \pi_2)
      + \frac{\pi_2}{r_{12}^2|\xi + \pi_2 - 1|^3}(\xi + \pi_2 - 1)
      - \frac{\xi}{r_{12}^2}\\
  0 & = \frac{(1 - \pi_2)}{|\xi + \pi_2|^3}(\xi + \pi_2) + \frac{\pi_2}{|\xi
      + \pi_2 - 1|^3}(\xi + \pi_2 - 1) - \xi\\
  f(\xi) & = \frac{(1 - \pi_2)}{|\xi + \pi_2|^3}(\xi + \pi_2)
           + \frac{\pi_2}{|\xi + \pi_2 - 1|^3}(\xi + \pi_2 - 1) - \xi
           \eqnumtag\label{fofxi}
\end{align*}
The roots of \cref{fofxi} are the Lagrange points \(L_1\), \(L_2\) and \(L_3\).

%%% Local Variables:
%%% mode: latex
%%% TeX-master: t
%%% End:
