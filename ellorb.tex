\hypersetup{pageanchor = false}
\section{Elliptical Orbit}
\label{elliptical}

For \(0 < e < 1\), we can try to integrate \cref{orbitalintegral} or use Kepler's method of the inscribed ellipse in the circle.
\begin{figure}
  \centering
  \includestandalone[mode = image]{ellipseincircle}
  \caption[Ellipse Inscribed in a Circle]{Ellipse inscribed in a circle.}
  \label{ellipseincircle}
\end{figure}

\hypersetup{pageanchor = false}
\subsection{Method 1: Kepler's Method}
\label{method-1:-keplers}

Let the area of the ellipse be \(A_0 = \pi ab\), the green shaded area be
\(A_1\), and the blue triangular area be \(A_2\).
The equation for an ellipse is \(\frac{x^2}{a^2} + \frac{y^2}{b^2} = 1\).
If we solve for \(y\), we have
\[
y = \frac{b}{a}\sqrt{a^2 - x^2}.
\]
Now, we can write \(\frac{A_1}{A_0} = \frac{t - t_0}{T}\).
The area of \(A_1 = A_{SVP} - A_2\) where \(A_2 = \frac{1}{2}\) base times
height.
The base can be found by taking \(ae - a\cos(E) = \text{base}\) and the
height is \(b\sin(E) = \text{height}\).
Therefore, \(A_2\) can be expressed as
\[
A_2 = \frac{ab}{2}[e\sin(E) - \cos(E)\sin(E)].
\]
Now let's consider the area \(A_{SVP}\).
We can write the area as \(A_{SVP} = \frac{b}{a}A_{QVP}\) where
\begin{align*}
  A_{QVP} & = A_{QOP} - A_{QOV}\\
          & = \frac{1}{2}a^2 E - \frac{1}{2}a^2\cos(E)\sin(E).
\end{align*}
Therefore, \(A_{SVP} = \frac{ab}{2}\bigl[E - \sin(E)\cos(E)\bigr]\) and
\(A_1 = \frac{ab}{2}\bigl[E - e\sin(E)\bigr]\).
Recall that \(\frac{A_1}{A_0} = \frac{t - t_0}{T}\) so
\[
\frac{t - t_0}{T} = \frac{1}{2\pi}\bigl[E - e\sin(E)\bigr]
\]
where \(T = \frac{2\pi}{\sqrt{\mu}}a^{3/2}\).
\[
\sqrt{\frac{\mu}{a^3}}(t - t_0) = E - e\sin(E) \eqnumtag\label{keplerseqell}
\]
where \(n = \sqrt{\frac{\mu}{a^3}}\); that is, the mean motion
\(M_e = n(t - t_0) = E - e\sin(E)\) and \(E\) is the eccentric anomaly.

\hypersetup{pageanchor = false}
\subsection{Method 2: Integration}
\label{meth-2:-integr}

We can always try to directly integrate the function.
\[
\frac{\mu^2}{h^3}t = \int_0^{\nu}\frac{\ud\nu'}{(1 + e\cos(\nu'))^2}
\eqnumtag\label{orbintell}
\]
Consider
\begin{align*}
  \frac{d}{d\nu'}\frac{\sin(\nu')}{1 + e\cos(\nu')}
  & = \frac{\cos(\nu') + e}{(1 + e\cos(\nu'))^2}\\
  \intertext{Then}
  \frac{d}{d\nu'}\frac{e\sin(\nu')}{1 + e\cos(\nu')}
  & = \frac{1}{1 + e\cos(\nu')} + \frac{e^2 - 1}{(1 + e\cos(\nu'))^2}.
\end{align*}
We can now isolate a somewhat easier integral.
\[
\int_0^{\nu}\frac{\ud\nu'}{(1 + e\cos(\nu'))^2} =
\frac{e}{e^2 - 1}\frac{\sin(\nu)}{1 + e\cos(\nu)} -
\frac{1}{e^2 - 1}\int_0^{\nu}\frac{\ud\nu'}{1 + e\cos(\nu')}
\eqnumtag\label{int1}
\]
After integrating \cref{int1}, we end up with
\[
\int_0^{\nu}\frac{\ud\nu'}{(1 + e\cos(\nu'))^2} =
\frac{e}{e^2 - 1}\frac{\sin(\nu)}{1 + e\cos(\nu)} -
\frac{2}{(e^2 - 1)^{3/2}}\arctan\biggl[\sqrt{\frac{e - 1}{e + 1}}
\tan\Bigl(\frac{\nu}{2}\Bigr)\biggr] = \frac{\mu^2}{h^3}t
\eqnumtag\label{int2}
\]
and after simplifying \cref{int2}, we have
\[
\frac{\mu^2}{h^3}(1 - e^2)^{3/2}t =
2\arctan\biggl[\sqrt{\frac{1 - e}{1 + e}}
\tan\Bigl(\frac{\nu}{2}\Bigr)\biggr] - \frac{e\sqrt{1 - e^2}
  \sin(\nu)}{1 + e\cos(\nu)} = M_e.
\eqnumtag\label{int3}
\]
\begin{laws}{Kepler's \(3^{\text{rd}}\) Law}
  The square of the orbital period of a planet is proportional to the cube
  of the semi-major axis of its orbit.
\end{laws}
That is, \(T = \frac{2\pi}{\sqrt{\mu}}a^{3/2}
= \frac{2\pi}{\mu^2}\bigl(\frac{h}{\sqrt{1 - e^2}}\bigr)^3\).
So \(M_e = \left(\frac{2\pi}{T}\right)t = nt\).
Referring back to \cref{ellipseincircle}, we have from the
geometry that \(a\cos(E) = ae + r\cos(\nu)\) where
\(r = \frac{p}{1 + e\cos(\nu)} = \frac{a(1 - e^2)}{1 + e\cos(\nu)}\).
\begin{align*}
  a\cos(E)
  & = ae + \frac{a(1 - e^2)}{1 + e\cos(\nu)}\\
  \cos(E)
  & = \frac{e + \cos(\nu)}{1 + e\cos(\nu)}
\end{align*}
Here we can solve for \(\cos(\nu)\).
\[
\cos(\nu) = \frac{e - \cos(E)}{e\cos(E) - 1}
\]
Unfortunately, \(\cos(\nu)\) is multi-valued for \(\nu\in [0, 2\pi]\).
Consider \(\tan^2\bigl(\frac{E}{2}\bigr)
= \frac{\sin^2\bigl(\frac{E}{2}\bigr)}{\cos^2\bigl(\frac{E}{2}\bigr)}\).
Using the power rule for sine and cosine, we have
\[
\tan^2\Bigl(\frac{E}{2}\Bigr) = \frac{1 - \cos(E)}{1 + \cos(E)}.
\]
Next write
\(1 - \cos(E) = \frac{1 + \cos(\nu)}{1 + \cos(\nu)} -
\frac{e + \cos(\nu)}{1 + \cos(\nu)} =
\frac{(1 - e)(1 - \cos(\nu))}{1 + e\cos(\nu)}\) and
\(1 + \cos(E) = \frac{(1 + e)(1 + \cos(\nu))}{1 + e\cos(\nu)}\).
Then
\[
\tan^2\Bigl(\frac{E}{2}\Bigr) =
\frac{1 - e}{1 + e}\underbrace{\frac{1 - \cos(\nu)}{1 +
    \cos(\nu)}}_{\tan^2(\frac{\nu}{2})}.
\]
Thus,
\(E = 2\arctan\Bigl[\sqrt{\frac{1 - e}{1 + e}}
\tan\bigl(\frac{\nu}{2}\bigr)\Bigr]\) which is the first term in Kepler's
method.
As for the \(\sin(E)\) term,
\begin{align*}
  \sin(E) & = \sqrt{1 - \cos^2(E)}\\
          & = \frac{\sqrt{1 - e^2}\sin(\nu)}{1 + e\cos(\nu)}
\end{align*}
which is exactly what we needed.
\begin{figure}
  \centering
  \subcaptionbox{The mean anomaly as a function of the eccentric anomaly.}{
    \includegraphics[width = 2.5in, trim = 20bp 20bp 20bp 40bp,
    clip]{eccenmeananomfunc}}
  \qquad
  \subcaptionbox{The mean anomaly as a function of the true anomaly.}{
      \includegraphics[width = 2.5in, trim = 20bp 20bp 20bp 40bp,
      clip]{eccentrueanomfunc}}
  \caption[Mean Anomaly Function Ellipse]
  {The mean anomaly as a function of the eccentric and true anomalies,
    respectively (\(e < 1\)).}
  \label{meananomalyfunctionofeccentric}
\end{figure}
%%% Local Variables:
%%% mode: latex
%%% TeX-master: t
%%% End:
