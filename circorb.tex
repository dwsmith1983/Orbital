\hypersetup{pageanchor = false}
\section{Circular Orbit}
\label{circular}

For \(e = 0\), \cref{orbitalintegral} becomes
\begin{align}
  \frac{\mu^2}{h^3}\int_0^t\ud t' & = \int_0^{\nu}\ud\nu'\notag\\
  \frac{\mu^2}{h^3}t & = \nu\label{orbintcirc}.
\end{align}
Solving for \(t\) in \cref{orbintcirc}, we have \(t = \frac{h^3\nu}{\mu^2}\).
For a circular orbit, \(r = \frac{h^2}{\mu}\).
Then \(r^{3/2} = \frac{h^3}{\mu^{3/2}}\).
Now we can solve for \(h\) and substitute into \(t\).
\[
t = \frac{r^{3/2}\nu}{\sqrt{\mu}}
\]
By Kepler's \(3^{\text{rd}}\) Law, if \(\nu = 2\pi\), then \(t = T\), the
orbital period.
Therefore, \(t = \frac{\nu}{2\pi}T\) so \(\nu = 2\pi\frac{t}{T}\).
Let \(2\pi\frac{t}{T} = nt\) where \(n\) is the mean motion.
Then \(\nu(t) = nt\).
%%% Local Variables:
%%% mode: latex
%%% TeX-master: t
%%% End:
