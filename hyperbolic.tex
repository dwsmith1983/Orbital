\chapter{Hyperbolic Trajectories}
\label{hyperb-traj}

\begin{figure}
  \centering
  \includestandalone[mode = image]{geometryhyperbola}
  \caption[Geometry of a Hyperbola]{The geometry of a hyperbolic trajectory.}
  \label{geometryhyperbola}
\end{figure}
\begin{alignat*}{4} 
  \nu_{\infty} & = \text{The true anomaly} & \qquad & \Delta
  &&{}= a\sqrt{e^2 - 1} \text{ The aiming radius}\\
  \delta & = 2\sin^{-1}\Bigl(\frac{1}{e}\Bigr) \text{ The turn angle} & & c
  &&{}= \text{Distance from } F \text{ to } F_2\\
  e & > 1 & & a &&{}= \frac{h^2}{\mu(e^2 - 1)}\\
  \mathcal{E} & > 0 & & v_{\infty} &&{}= \frac{\mu}{h}\sqrt{e^2 - 1}
  \text{ The hyperbolic excess speed}
\end{alignat*}
See \cref{geometryhyperbola}.
\begin{examples}{Hyperbolic Trajectories}{hyperbolictrajectories}
  A meteoroid on a hyperbolic orbit is first observed approaching the Earth when
  it is 402,000 km from the center of the Earth with a true anomaly of
  \(150^{\circ}\).
  If the speed of the meteoroid at that time is 2.23 km/s:
  \par\smallskip
  \begin{minipage}{\linewidth}
    \centering
    \includestandalone[mode = image]{incomingmeteroid}
    \captionof{figure}[Incoming Meteor]
    {An incoming hyperbolic trajectory approaching Earth.}
    \label{incomingmeteoroid}
  \end{minipage}
  \begin{enumerate}[label = (\alph*)]
  \item
    Calculate the eccentricity of the trajectory
    \par\smallskip
    The energy for the orbit is
    \begin{align*} 
      \mathcal{E} & = \frac{v^2}{2} - \frac{\mu}{r}\\
                  & = \frac{2.23^2}{2} - \frac{398600}{402000}\\
                  & = 2.48645 - 0.991542\\
                  & = 1.4949\text{km}^2/\text{s}^2
    \end{align*}
    We can write \(a = \frac{h^2}{\mu(1 - e^2)}\Rightarrow h^2 =
    a\mu(1 - e^2)\) and \(\mathcal{E} = -\frac{\mu}{2a}\).
    \begin{align*} 
      h^2 & = a\mu(1 - e^2)\\
          & = \frac{\mu^2(e^2 - 1)}{2\mathcal{E}}\\
          & = \frac{398600^2(e^2 - 1)}{2\times 1.4949}\\
          & = 5.31413\times 10^{10}(e^2 - 1)
    \end{align*}
    The orbital equation is
    \(r = \frac{h^2}{\mu(1 + e\cos\nu)}\Rightarrow h^2 = r\mu(1 + e\cos\nu)\).
    \begin{align*} 
      h^2 & = r\mu(1 + e\cos\nu)\\
          & = 402000\cdot 3986000\biggl[1 + e
            \cos\Bigl(\frac{5\pi}{6}\Bigr)\biggr]\\
          & = 1.60237\times 10^{11}\biggl(1 - \frac{e\sqrt{3}}{2}\biggr)
    \end{align*}
    Now, we can equate the equations for \(h^2\) and solve for \(e\).  
    \begin{align*} 
      5.31413\times 10^{10}(e^2 - 1)
      & = 1.60237\times 10^{11}\biggl(1 - \frac{e\sqrt{3}}{2}\biggr)\\
      e^2 - 1 & = 3.0153 - \frac{3.01533e\sqrt{3}}{2}\\
      e^2 + 2.61133e - 4.0153 & = 0\\
      e & = \frac{-2.61133\pm\sqrt{2.61133^2 + 4\cdot 4.0153}}{2}\\
      e & = -3.69731, 1.08601
    \end{align*}
    Since \(e \geq 0\), the eccentricity of the orbit is \(e = 1.08601\).
  \item
    Calculate the altitude at closest approach
    \par\smallskip
    We can use \(e\) from (a) to solve for \(h^2\).
    \begin{align*} 
      h^2 & = r\mu(1 + e\cos\nu)\\
          & = 402000\cdot 398600\biggl(1 - \frac{1.08601\sqrt{3}}{2}\biggr)\\
          & = 97639.155\text{ km}^2/\text{s}^2\
    \end{align*}
    Now
    \begin{align*} 
      r_p & = \frac{h^2}{\mu(1 + e)}\\
          & = \frac{97639.155}{398600\cdot 2.08601}\\
          & = 11465.6\text{km}
    \end{align*}
    So the altitude from the Earth is \(r_p - r_e\).
    \[
    11465.6 - 6378 = 5087.59\text{ km}
    \]
  \item
    Calculate the speed at closest approach
    \par\smallskip
    Velocity at closet approach can be found from
    \begin{align*} 
      v & = \frac{h}{r_p}\\
        & = \frac{\sqrt{97639.155}}{11464}\\
        & = 8.5158\text{ km}/\text{s}
    \end{align*}
  \item
    Plot the trajectory to scale
    \par\smallskip
    \begin{minipage}{\linewidth}
      \centering
      \includegraphics[width = 3.5in, height = 3.5in, trim = 88bp 15bp 95bp
      30bp, clip]{example3notes}
      \captionof{figure}[Meteor on Hyperbolic Flyby]
      {Meteorite on a hyperbolic orbit near Earth.}
      \label{hyperbolicnearearth}
    \end{minipage}
  \item
    Calculate the hyperbolic excess speed
    \par\smallskip
    The equation for the hyperbolic excess speed is
    \begin{align*} 
      v_{\infty} & = \frac{\mu}{h}\sqrt{e^2 - 1}\\
                 & = \frac{398600}{\sqrt{97639.155}}\sqrt{1.08601^2 - 1}\\
                 & = 1.72932\text{ km}/\text{s}
    \end{align*}
  \end{enumerate}
  Using a scientific software package, to develop a computer program to
  compute the trajectory of meteorite from the equation of motion
  \begin{align*} 
    \ddot{\mathbf{r}} + \mu\frac{\mathbf{r}}{r^3} & = 0
  \end{align*}
  \par\smallskip
  \begin{minipage}{\linewidth}
    \centering
    \includegraphics[width = 3.5in, height = 3.5in, trim = 70bp 15bp 90bp 30bp,
    clip]{example3notes2}
    \captionof{figure}[Simulated Hyperbolic Trajectory]
    {The trajectory using the 3 body numerical simulation.}
    \label{numsimulationhyperbolic}
  \end{minipage}
  \par\smallskip
  Refer to Python file example3notes2.py: where \(r(0) = rx\), \(r(1) = ry\),
  and \(r2(0) = 0\) are the initial position vector, \(v(0) = vx\),
  \(v(1) = vy\), and\(v(2) = 0\) are the initial velocity vector, and
  \(\mathbf{r}\) and \(\mathbf{v}\) were found from
  \begin{align*} 
    \mathbf{r} & = r\cos(\nu)\hat{\mathbf{p}} + r\sin(\nu)\hat{\mathbf{q}}\\
    \mathbf{v} & = \frac{\mu}{h}\Bigl[\sin(\nu)\hat{\mathbf{p}} - (e
                 + \cos(\nu))\hat{\mathbf{q}}\Bigr]
  \end{align*}
\end{examples}
\begin{examples}{Hyperbolic Venus Fly By}{hyperbolicflyby}
  An interplanetary probe intended for the outer solar system is launched
  from   Earth.
  Initially it is sent towards Venus on a flyby trajectory in order to perform
  a gravity assist maneuver see \cref{venusflyby}.
  Referring to the diagram below, suppose the spacecraft enters the sphere of
  influence of Venus (\(r_{\text{SOI}} = 616000\) km) with a heliocentric
  speed of \(37.7\) km/s and a heliocentric flight path angle
  \(\sigma_2 = 20^{\circ}\).
  Suppose that the hyperbolic trajectory asymptote is designed with an aiming
  radius of \(\Delta = 1.5\) planetary radii (\(r_{\Venus} = 6,051.8\) km).
  For this problem, the heliocentric velocity of Venus is 35.022km/s and the 
  gravitational parameter for Venus is
  \(\mu_{\Venus} = 324859\text{km}^3/\text{s}^2\).
  \par\smallskip
  \begin{minipage}{\linewidth}
    \centering
    \includegraphics[width = 3.5in]{venusflyby}
    \captionof{figure}[Flyby of Venus]{Depiction of the flyby of Venus.}
    \label{venusflyby}
  \end{minipage}
  \begin{enumerate}[label = (\alph*)]
  \item
    What is the arrival/departure speed \(v_2\) and \(v_3\) relative to Venus?
    \par\smallskip
    \begin{minipage}{\linewidth}
      \centering
      \includestandalone[mode = image]{geoapproach}
      \captionof{figure}[Geometry of Approach]
      {Close up of the geometry of the incoming spacecraft.}
      \label{geoapproach}
    \end{minipage}
    \par\smallskip
    The speed in the x direction is \(\mathbf{v}_x = 37.7\cos\frac{\pi}{9}\)
    km/s but this isn't the speed relative to Venus see \cref{geoapproach}.
    \begin{align*} 
      \mathbf{v}_{x\text{rel}} & = 37.7\cos\frac{\pi}{9} - 35.022\\ 
                               & = 0.404412\text{ km}/\text{ s}
    \end{align*}
    The speed in the y direction relative to Venus is
    \(\mathbf{v}_y = -37.7\sin\frac{\pi}{9}\) so the relative arrival velocity
    vector to Venus is
    \begin{align*} 
      \mathbf{v}_2
      & =  v_x\hat{\mathbf{i}} + v_y\hat{\mathbf{j}} + v_z\hat{\mathbf{k}}\\
      & = 0.404412\hat{\mathbf{i}} - 12.8942\hat{\mathbf{j}} +
        0\hat{\mathbf{k}}
    \end{align*}
    So the relative arrival speed to Venus is
    \(\lVert\mathbf{v}_2\rVert = \sqrt{0.404412^2 + 12.8942^2} = 12.9005\)
    km/s which is also the departure speed.
  \item
    What is the eccentricity of the fly-by trajectory?
    \par\smallskip
    From the energy equation, we have
    \begin{align*} 
      \mathcal{E} & = \frac{v_2^2}{2}
                    - \frac{\mu_{\Venus}}{\Venus\text{SOI}}\\
                  & = \frac{12.9005^2}{2} - \frac{324859}{616000}\\
                  & = 82.6841
    \end{align*}
    Now we can use \(\mathcal{E} = \frac{\mu_{\Venus}}{2a}\), to solve for the 
    semi-major axis, \(a\).
    \begin{align*} 
      a & = \frac{\mu_{\Venus}}{2\mathcal{E}} = 1964.46\text{ km}.
    \end{align*}
    Finally, we can use the equation for the aiming radius to the
    eccentricity, \(e\).
    \begin{align*} 
      \Delta & = 1.5\cdot 6051.8\\
             & = 9077.7\\
      9077.7 & = a\sqrt{e^2 - 1}\\
      e & = \sqrt{\Bigl(\frac{9077.7}{1964.46}\Bigr)^2 + 1}\\
             & = 4.72793
    \end{align*}
  \item
    What is the magnitude of the heliocentric velocity as it leaves the sphere
    of influence? What percentage increase in heliocentric velocity is
    obtained?
    \par\smallskip
    The magnitude of the heliocentric velocity is 
    \[ 
    v_{3\Sun} = \sqrt{35.022^2 + v_2^2 + 2v_2\cdot 35.022
      \cos(\pi - \delta - \theta_2)}\text{ km}/\text{s}
    \] 
    where \(\theta_2 = \sigma_2 + \varphi_2\) and
    \(\delta = 2\sin^{-1}\bigl(\frac{1}{e}\bigr)\).
    \begin{align*} 
      \delta & = 2\sin^{-1}\Bigl(\frac{1}{e}\Bigr)\\
             & = 0.426237\text{ rad}\quad\text{and}\quad\\
      \varphi_2 & = \cos^{-1}\Bigl(\frac{\mathbf{v}_2\cdot
                  \mathbf{v}_{\Venus}}{v_2v_{\Venus}}\Bigr) - \frac{\pi}{9}\\
             & = 1.19038\text{ rad}
    \end{align*}
    Therefore, \(v_{3\Sun} = 41.7203\) km/s.
    The percentage increase is
    \[
    \frac{41.7203 - 37.7}{37.7}\times 100\% = 10.6638\%.
    \]
  \end{enumerate}
\end{examples}
%%% Local Variables:
%%% mode: latex
%%% TeX-master: t
%%% End:
