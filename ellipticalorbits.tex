\hypersetup{pageanchor = false}
\chapter{Elliptical Orbit}
\label{elliptical-orbit}

Elliptical orbits are orbits in which the functional form is
\(\frac{x^2}{a^2} + \frac{y^2}{b^2} = 1\) or can be written as a vector valued
function of the form
\(\mathbf{r}(t) = a\cos(t)\mathbf{i} + b\sin(t)\mathbf{j}\).
\begin{enumerate}[label = \arabic*.]
\item
  Conservation of Mechanic Energy:
  \(\mathcal{E} = \dfrac{v^2}{2} - \dfrac{\mu}{r} < 0\)
\item
  Conservation of Angular Momentum:
  \(\mathbf{h} = \mathbf{r}\times\mathbf{v}\)
\item
  The Trajectory Equation: \(r = \dfrac{h^2}{\mu(1 + e\cos(\nu))}\)
\end{enumerate}
\begin{figure}
  \centering
  \includestandalone[mode = image]{geometryellipse}
  \caption[Geometry of an Ellipse]{The geometry of an ellipse.}
  \label{geometryellipse}
\end{figure}
From the \cref{geometryellipse}, we that 
\begin{enumerate}[resume*]
\item
  \(p\) is the semi-latus rectum,
\item
  the distance from the focus to \(a\) is \(r_p\) (radius at periapsis),
\item
  the distance from the origin to \(c\) is \(ae\) where \(e\) is the
  eccentricity, and
\item
  \(0 < e = 1 - \frac{c}{a} < 1\).
\end{enumerate}
We can then write \(r = \frac{p}{1 + e\cos(\nu)}\) and \(p = a(1 - e^2)\).
Moreover, \(p = \frac{h^2}{\mu}\).
At periapsis, \(\nu = 0\), and at apoapsis, \(\nu = \pi\) so
\begin{align}
  r_p & = \frac{p}{1 + e} = a(1 - e)\\
  r_a & = \frac{p}{1 - e} = a(1 + e).\label{rpra}
\end{align}
Now we can take the ratio of \(r_p\) and \(r_a\).
\begin{align} 
  \frac{r_p}{r_a} & = \frac{1 - e}{1 + e}\notag\\
  e & = \frac{r_a - r_p}{r_a + r_p}\label{eccinrpra}
\end{align}
Let's examine the specific energy at periapsis.
Then \(v = v_p\) and \(r = r_p\).
Recall that \(h = rv\sin\theta\).
At periapsis, the angle between \(r\) and \(v\) is \(\theta = \frac{\pi}{2}\)
so \(h = rv_p\Rightarrow v_p = \frac{h}{r_p}\).
Now we can write the specific energy as 
\begin{align*}
  \mathcal{E} & = \frac{v^2}{2} - \frac{\mu}{r}\\
              & = \frac{v_p^2}{2} - \frac{\mu}{r_p}\\
              & = \frac{h^2}{2r_p^2} -
                \frac{\mu}{r_p}\eqnumtag\label{energyatrp}
\end{align*}
Using \cref{energyatrp} and the relation that \(p = \frac{h^2}{\mu}\) at periapsis, we have \(h^2 = pu\) but \(p = a(1 - e^2)\) so
\(h^2 = a(1 - e^2)u\).
\begin{align*}
  \mathcal{E} & = \frac{h^2}{2r_p^2} - \frac{\mu}{r_p}\\
              & = \frac{a(1 - e^2)\mu}{2a^2(1 - e)^2} -
                \frac{\mu}{a(1 - e)}\\
              & = \frac{(1 - e^2)\mu - 2\mu(1 - e)}{2a(1 - e)^2}\\
              & = \frac{\mu(1 + e - 2)}{2a(1 - e)}\\
              & = \frac{\mu(e - 1)}{2a(1 - e)}\\
              & = -\frac{\mu}{2a}\eqnumtag\label{energyform2}
\end{align*}
Recall \cref{previousintegrated}.
Then \(\mu\mathbf{e} = (\dot{\mathbf{r}}\times\mathbf{h})
- \mu\frac{\mathbf{r}}{r}\).
\begin{align*}
  \mu\mathbf{e} & = \dot{\mathbf{r}}\times\mathbf{h} -
                  \mu\frac{\mathbf{r}}{r}\\
                & = \dot{\mathbf{r}}\times\mathbf{r}\times\dot{\mathbf{r}}
                  - \mu\frac{\mathbf{r}}{r}\\
                & = v^2\mathbf{r} - (rv)\mathbf{v} - \mu\frac{\mathbf{r}}{r}\\
                & = \left(v^2 - \frac{\mu}{r}\right)\mathbf{r} - (rv)\mathbf{v}
                  \eqnumtag\label{eccenvec}
\end{align*}
At this point in the notes, given \(\mathbf{r}\) and \(\mathbf{v}\), we are able to find
\begin{enumerate}[label = \Roman*.] 
\item
  \(\mathcal{E} = \frac{v^2}{2} - \frac{\mu}{r} = -\frac{\mu}{2a}\)
\item
  \(\mathbf{e} = \frac{1}{\mu}\left[\left(v^2 - \frac{\mu}{r}\right)\mathbf{r}
    - (rv)\mathbf{v}\right]\) and \(e = \lVert\mathbf{e}\rVert\).
\item
  \(p = a(1 - e^2)\Rightarrow r = \frac{p}{1 + e\cos\nu}\) so we can also find
  the true anomaly \(\nu\).
\item
  \(\mathbf{h} = \mathbf{r}\times\mathbf{v}\)
\end{enumerate}
\begin{figure}
  \centering
  \includestandalone[mode = image]{componentsvelocity}
  \caption[Components of Velocity]
  {The components of the velocity vector \(\mathbf{v}\) consist of the
    perpendicular and the parallel velocities.}
  \label{componentsvelocity}
\end{figure}
\noindent
In the diagram above,
\(\lVert\mathbf{v}_r\rVert = \lVert\mathbf{v}_{\parallel}\rVert\).
\begin{alignat*}{4}
  v_{\perp} & = \frac{h}{r} &\quad & v_r &&{} = \dot{r}\\
  & = \frac{\mu}{h}(1 + e\cos(\nu)) &\quad &
  &&{} = \frac{d}{dt}\left[\frac{h^2}{\mu(1 + e\cos(\nu))}\right]\\
  & & & &&{} = -\frac{h^2e\dot{\nu}\sin(\nu)}{\mu(1 + e\cos(\nu))^2}\\
  & & & &&{} = -\frac{h}{r^2}\frac{h^2e\sin(\nu)}{\mu(1 + e\cos(\nu))^2}\\
  & & & &&{} = -\frac{\mu^2(1 + e\cos(\nu))^2}{h^4}
  \frac{h^3e\sin(\nu)}{\mu(1 + e\cos(\nu))^2}\\
  & & & &&{} = \frac{\mu e\sin(\nu)}{h}
\end{alignat*}
We were able to make the substitution \(\dot{\nu} = \frac{h}{r^2}\).
Let \(\omega\) be the angular velocity.
Then \(\omega = r\frac{d\nu}{d t} = r\dot{\nu}\).
\(\omega\) is related to the tangential velocity, \(v_{\perp}\), by
\(v_{\perp} = r\dot{\nu}\).
\[ 
\tan(\gamma)  = \frac{v_r}{v_{\perp}} = \frac{e\sin(\nu)}{1 + e\cos(\nu)}
\eqnumtag\label{flightpath}
\]
Recall that \(\frac{dA}{dt} = \frac{1}{2}h\) and let \(T\) be the period of
the ellipse where \(A_0\) be the total area of the ellipse.
\begin{align} 
  \int_0^T \ud t & = \frac{2}{h}\int_0^{A_0}\ud A\notag\\
  T & = \frac{2A_0}{h}\label{periodellipinarea}
\end{align}
The area of an ellipse is \(A_0 = \pi ab\) and \(b = a\sqrt{1 - e^2}\).
Recall \(\frac{h^2}{\mu} = a(1 - e^2)\Rightarrow h = \sqrt{\mu a(1 - e^2)}\).
So the period of an ellipse is
\begin{align} 
  T & = \frac{2\pi a^2\sqrt{1 - e^2}}{\sqrt{\mu a(1 - e^2)}}\notag\\ 
  & = \frac{2\pi}{\sqrt{\mu}}a^{3/2}.\label{periodellip}
\end{align}

%%% Local Variables:
%%% mode: latex
%%% TeX-master: t
%%% End:
