\hypersetup{pageanchor = false}
\section{GEO Synchronous Earth Orbits}
\label{geo-synchr-earth}

First, we need to note the difference in a sidereal day and synodic day.
A sidereal day is the time it takes for the Earth to rotate \(360^{\circ}\) on
its axis, \(T = 23.93\) hours.
A synodic day is the time it takes for the Sun to appear in the same place
overhead, \(T = 24\) hours.
Now, for a circular orbit, \(T = \frac{2\pi}{\mu}r^{3/2}\).
To put a satellite in GEO around the Earth, \(T = 23.93\) hours where
\(\mu = G(m_{\Earth} + m_{\text{satellite}})\approx \mu_{\Earth} =
398600\frac{\text{km}^3}{\text{s}^2}\).
Now all we need to do is solve for \(r\).
\begin{align*} 
  r & = \left(\frac{23.93\cdot 3600\sqrt{398600}}{2\pi}\right)^{2/3}\\
    & = 42158.9\text{ km}
\end{align*}
The radius of the Earth is \(6378\) km so the altitude of a satellite in GEO
is \(35780.9\) km.
\begin{examples}{Elliptical Orbit}{ellipticalorbit}
  For an elliptical orbit show that the orbital speed as a function of the true
  anomaly is
  \[ 
  v = \frac{\mu}{h}\sqrt{e^2 + 2e\cos(\nu) + 1}.
  \]
  Plot the normalized speed \(v^* = \frac{vh}{\mu}\) as a function of the true  
  anomaly for different eccentricities between zero and unity.
  \smallskip

  First, we need to note that \(v_r = \frac{\mu}{h}e\sin(\nu)\) and
  \(v_{\perp} = \frac{\mu(1 + e\cos(\nu))}{h}\).
  \par\medskip
  \begin{minipage}{\linewidth}
    \centering
    \includestandalone[mode = image]{velcomp}
    \captionof{figure}[Example Components of Velocity]
    {Components of velocity vector example.}
    \label{velcomp}
  \end{minipage}
  \par\medskip
  By substitution, we have
  \begin{align*} 
    v & = \sqrt{v_r^2 + v_{\perp}^2}\\
      & = \sqrt{\left(\frac{\mu}{h}e\sin(\nu)\right)^2 
        + \left(\frac{\mu(1 + e\cos(\nu))}{h}\right)^2}\\
      & = \frac{\mu}{h}\sqrt{\left(e\sin(\nu)\right)^2 + (1 + e\cos(\nu))^2}\\
      & = \frac{\mu}{h}\sqrt{e^2(\sin^2(\nu) + \cos^2(\nu)) + 1 +
        2e\cos(\nu)}\\
      & = \frac{\mu}{h}\sqrt{e^2 + 2e\cos(\nu) + 1}
  \end{align*}
  For the plot of \(v^* = \frac{vh}{\mu} = \sqrt{e^2 + 2e\cos(\nu) + 1}\),
  see   \cref{velocityprofiles}.
  \par\medskip
  \begin{minipage}{\linewidth}
    \centering
    \includegraphics[width = 3.5in, height = 3.5in, trim = 40bp 20bp 45bp 35bp,
    clip]{example1notes}
    \captionof{figure}[Velocity Profiles]
    {Velocity profiles for \(\nu\in [0,2\pi]\) and \(e\in[0,1]\).}
    \label{velocityprofiles}
  \end{minipage}
\end{examples}

%%% Local Variables:
%%% mode: latex
%%% TeX-master: t
%%% End:
